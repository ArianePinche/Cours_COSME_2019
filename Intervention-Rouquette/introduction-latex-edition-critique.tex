\documentclass[transnotheorems,noamsthm]{beamer}
\usepackage{fontspec,polyglossia,xunicode,hyperref,csquotes}
\usepackage{graphicx}
\graphicspath{{img/}{}}
\usetheme[]{Boadilla}
\usepackage{metalogo}
\newenvironment{slide}{%
  \begin{frame}
  <presentation>\mode<presentation>{\frametitle{\insertsubsection}}%
  }%
  {\end{frame}}
\newenvironment{slide*}{\begin{frame}<presentation>[<1>]\mode<presentation>{\frametitle{\insertsubsection}}}{\end{frame}}

\AtBeginSection{
  \begin{frame}
    \sectionpage
  \end{frame}
}

\beamerdefaultoverlayspecification{<+->}
\setmainfont{Linux Libertine O}
\setmainlanguage{french}
\setotherlanguage{english}
\usepackage{minted}
\newcounter{code}
\resetcounteronoverlays{code}
\usepackage{hyperref}
\newcommand{\package}[1]{\emph{#1}}
\hypersetup{bookmarksdepth=6}
\NewDocumentCommand{\code}{mmo}{%
  \stepcounter{code}%
  \begin{block}{code \thecode: #2}%
			\tiny\IfNoValueF{#3}{#3}%
      \inputminted[linenos=true,breaklines=true, ]{latex}{code/#1}%
  \end{block}
	}
\setcounter{tocdepth}{1}
\setbeamertemplate{footline}[frame number]
\setbeamertemplate{navigation symbols}{}%remove navigation symbols
\newcommand{\shortcode}{\mintinline{latex}}
\usepackage[]{qrcode}
\usepackage[style=verbose,autocite=plain]{biblatex}

%%% les patch de reledmac et de biblatex sont incompatibles. En attendant on annule celui de reledmac.
\makeatletter
\let\old\@footnotetext
\usepackage[noeledsec,noend,series={A,B}]{reledmac}
\let\@footnotetext\old
\makeatother
\defbibheading{bibliography}{}
\addbibresource{biblio.bib}
\renewcommand{\newunitpunct}[0]{\addcomma\addspace}

\author{Maïeul Rouquette}
\title{Mettre en forme une édition critique avec \LaTeX}
\institute{Formation XSLT, Cosme, Lyon}
\date{23 avril 2019}
\newcommand{\linkintitlepage}[1]{%
  \bgroup
    \tiny
    \vfill
    \parbox[c][2cm]{0.8\textwidth}{
      \url{#1}
      \vfill
      Licence Creative Commons France 3.0 - Paternité - Partage à l'identique
    }
    \hfill \qrcode{#1}
  \egroup
}

\begin{document}


\begin{frame}
	\titlepage
  \linkintitlepage{https://geekographie.maieul.net/229}
\end{frame}

\begin{frame}
  \tableofcontents
\end{frame}
\section{Qu'est-ce que \LaTeX\ ?}

\subsection{Un logiciel de typographie}

\begin{slide}
  \begin{itemize}
    \item Un logiciel de typographie automatisé
    \item Un logiciel non Wysiwyg (\enquote{\textenglish{What you see is what you get}})
    \item L'utilisateur / utilisatrice saisit des commandes pour structurer sémantiquement son texte
    \item \LaTeX\ transforme ce code en PDF de haute qualité typographique
    \item \LaTeX\ est libre et gratuit
    \item \LaTeX\ existe en 3 déclinaisons principales : pdf\LaTeX, \XeLaTeX\ et Lua\LaTeX
  \end{itemize}
\end{slide}

\begin{slide}
  \centering
  \includegraphics[height=0.8\textheight]{comparison.pdf}

  \url{http://www.rtznet.nl/zink/latex.php?lang=nl}
\end{slide}
\subsection{Un logiciel extensible}

\begin{slide}
  \begin{itemize}
    \item Il est possible de créer ses propres commandes \LaTeX\ pour automatiser certaines tâches
    \item De nombreux packages pour gérer des problèmes spécifiques:
      \begin{itemize}
        \item Bibliographie : \package{Bib\LaTeX}
        \item Index : \package{indextools}
        \item Dessin vectoriel : \package{TikZ}
        \item Édition critique et traduction parallèle : \package{reledmac} et \package{reledpar} (anciennement \package{ledmac} / \package{ledpar})
          \item etc.
      \end{itemize}
  \end{itemize}
\end{slide}
\section{\LaTeX\ pour les éditions critiques}
\subsection{Ce pourquoi \LaTeX\ n'est pas prévu}

\begin{slide}
  \begin{itemize}
    \item Pour collationner les manuscrits
    \item Pour analyser les variantes et construire un \emph{stemma codicum}
    \item Pour établir le texte édité
    \item Pour avoir un encodage structurée de l'édition et obtenir plusieurs sorties
  \end{itemize}
\end{slide}
\subsection{Ce que permet le package \emph{reledmac} de \LaTeX}

\begin{slide}
  \begin{itemize}
    \item Mettre en page l'édition critique, avec notamment renvoi aux numéros de lignes
    \item Disposer de plusieurs niveaux de notes
    \item Configurer l'apparence des notes
    \item Mettre en forme de la poésie
    \item Mettre en parallèle l'édition et la traduction, en synchronisant automatiquement les pages
    \item Gérer l'apparat des manuscrits
    \item Le manuel fait plus de 100 pages\ldots
  \end{itemize}
\end{slide}

\begin{slide}
  \alert{$\Longrightarrow$}  \LaTeX\  est une sortie idéale d'une transformation XLST pour qui veut produire une forme \enquote{livre} de son édition critique
\end{slide}
\subsection{Principes généraux}

\begin{slide}
  \begin{itemize}
    \item L'édition critique est un \alert{texte numéroté}
    \item Il peut contenir des \alert{lemmes}
    \item Chaque lemme est associé à une ou plusieurs \alert{notes de bas de page}, qui se référent au numéro de ligne du lemme
    \item Il peut y avoir plusieurs niveaux de notes
    \item \package{reledmac} est \alert{neutre} quand au contenu de ces  notes. L'usage dépend du projet:
        \begin{itemize}
          \item Variantes
          \item Commentaires
          \item Apparat des sources
          \item etc.
        \end{itemize}
    \item  On dispose également de :
      \begin{itemize}
        \item Notes \enquote{familières} (par appel de note)
        \item Notes critiques de fin
        \item Notes de marge
      \end{itemize}
    \item \alert{$\Longrightarrow$} Le remplissage de ces notes fera partie des réglages de la feuille de style XLST
  \end{itemize}
\end{slide}

\section{Exemples}

\subsection{Exemple minimum}

\begin{slide}
  \code{reledmac_mwe.tex}{Un texte, quelques lemmes, des notes critiques sur un niveau}
  \begin{minipage}{0.8\textwidth}
    \beginnumbering
\pstart
Le petit \edtext{chat}{\Afootnote{A : chien}} est \edtext{mort}{\Afootnote{B : décédé}}.
Il est tombé du toit.
Pourquoi est-ce \edtext{toujours}{\Afootnote{C : \emph{om.}}} un petit chat qui meurt et jamais un pape qui tombe du  \edtext{toit}{\Afootnote{AD : \emph{add.} dans la rue}} ?
\pend
\endnumbering

  \end{minipage}
\end{slide}


\subsection{Un meilleur encodage}

\begin{slide}
  \code{reledmac_semantic_annotation.tex}{Commandes sémantiques}
  \begin{minipage}{0.8\textwidth}
    \newcommand{\variant}[3]{\edtext{#1}{\Afootnote{#2: #3}}}
\newcommand{\om}[2]{\variant{#1}{#2}{\emph{om.}}}
\newcommand{\add}[3]{\variant{#1}{#2}{\emph{add.} #3}}


\beginnumbering
\pstart
Le petit \variant{chat}{A}{chien} est \variant{mort}{B}{décédé}.
Il est tombé du toit.
Pourquoi est-ce \om{toujours}{C} un petit chat qui meurt et jamais un pape qui tombe du  \add{toit}{AD}{dans la rue} ?
\pend
\endnumbering

  \end{minipage}
\end{slide}

\subsection{Apparence des notes critiques}

\Xarrangement{paragraph}
\Xnumberonlyfirstinline
\Xsymlinenum{$||$}

\begin{slide}
  \code{reledmac_settings.tex}{Notes un peu plus compactes}
  \begin{minipage}{0.8\textwidth}
    \newcommand{\variant}[3]{\edtext{#1}{\Afootnote{#2: #3}}}
\newcommand{\om}[2]{\variant{#1}{#2}{\emph{om.}}}
\newcommand{\add}[3]{\variant{#1}{#2}{\emph{add.} #3}}


\beginnumbering
\pstart
Le petit \variant{chat}{A}{chien} est \variant{mort}{B}{décédé}.
Il est tombé du toit.
Pourquoi est-ce \om{toujours}{C} un petit chat qui meurt et jamais un pape qui tombe du  \add{toit}{AD}{dans la rue} ?
\pend
\endnumbering

  \end{minipage}
\end{slide}
\subsection{Lemmes longs et imbriqués}

\begin{slide}
  \code{reledmac_long_lemme.tex}{Un long lemme peut être abrégé}
  \begin{minipage}{0.8\textwidth}
    \newcommand{\variant}[4][]{%
  \edtext{#2}{%
    \ifstrempty{#1}{}{\lemma{#1}}%
    \Afootnote{#3: #4}%
  }%
}
\newcommand{\om}[3][]{\variant[#1]{#2}{#3}{\emph{om.}}}
\newcommand{\add}[4][]{\variant[#1]{#2}{#3}{\emph{add.} #4}}

\beginnumbering
\pstart
\om[Le petit \ldots\ du toit]{Le petit \variant{chat}{A}{chien} est \variant{mort}{B}{décédé}.
Il est tombé du toit.}{E}
Pourquoi est-ce  \om{toujours}{C} un petit chat qui meurt et jamais un pape qui tombe du  \add{toit}{AD}{dans la rue} ?
\pend
\endnumbering

  \end{minipage}
\end{slide}
\subsection{Mots identiques}

\newcommand{\variant}[3]{\edtext{#1}{\Afootnote{#2: #3}}}
\newcommand{\om}[2]{\variant{#1}{#2}{\emph{om.}}}
\newcommand{\add}[3]{\variant{#1}{#2}{\emph{add.} #3}}
\begin{slide}
  \code{reledmac_sameword.tex}{Mais de quel chat parle-t-on?}
  \begin{minipage}{0.8\textwidth}
    \beginnumbering
\pstart
Le petit \variant{\sameword{chat}}{A}{chien} est \variant{mort}{B}{décédé}.
Le \sameword{chat} est tombé du toit.
Pourquoi est-ce \om{toujours}{C} un petit \sameword{chat} qui meurt et jamais un pape qui tombe du  \add{toit}{AD}{dans la rue} ?
\pend
\endnumbering

  \end{minipage}
\end{slide}
\subsection{Textes et traductions en parallèles}

\begin{slide}
  \code{reledpar-pages.tex}{Exemple minimum avec \package{reledpar}}
\end{slide}
\section{Pour aller plus loin}

\subsection{Documentation}

\begin{slide}
  \nocite{*}
  \printbibliography
\end{slide}
\subsection{Installer \LaTeX}

\begin{slide}
   \begin{itemize}
     \item Installer une distribution \TeX : nous recommandons la \TeX Live (\url{http://tug.org/texlive/}) (y compris sous Windows)
     \item Possibilité de tester en ligne avec Overleaf (\url{https://www.overleaf.org})
   \end{itemize}
\end{slide}

\subsection{Interconnexion avec d'autres outils}
\begin{slide}
  \begin{itemize}
    \item L'Annexe B du manuel de \package{reledmac} fournit un certain nombre de liens vers des feuilles XLST existantes
    \item On pourra aussi commencer avec la TEI-Cat toolbox de Marjorie Burghart : \url{http://teicat.huma-num.fr/}
    \item Logiciel \package{samewords} (Michael Stenskjær Christensen)  \\
 \url{https://samewords.readthedocs.io/en/latest/}
  \end{itemize}
\end{slide}



\end{document}
