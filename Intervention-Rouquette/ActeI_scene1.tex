\documentclass{book}
\usepackage[noend,nofamiliar,series={A,B},noeledsec, noledgroup]{reledmac}
\usepackage{libertinus}
\usepackage{polyglossia}
\setmainlanguage{french}
\usepackage[paperwidth=10cm, paperheight=15cm]{geometry}

% Pas de numéro pour les chapitres, vers etc
\setcounter{secnumdepth}{-2}

% Réglage de la poésie
\setstanzaindents{2,0}%Indentation des vers, je n'ai aucune idée de s'il y a quelque chose spécifique à ce type de poésie
\setcounter{stanzaindentsrepetition}{1}
\sethangingsymbol{[\,}%rejet de vers
\newcommand{\antilabe}{\skipnumbering\unskip\hspace{2\stanzaindentbase}}

% Annonce des personnages
\newcommand{\personscene}[1]{\par\hspace{2\stanzaindentbase}\emph{#1}}
\newcommand{\enonciateur}[1]{\par\hspace{\stanzaindentbase}\textbf{#1}}
\begin{document}

\tableofcontents
\newpage
\beginnumbering
\stanza[\chapter{Acte premier}
  \section{Scene premiere}
  \personscene{Oreste, Pylade}
  \enonciateur{Oreste}
]
O VY, puis que ie retrouue vn Amy ſi fidelle,&
Ma Fortune va prendre vne face nou-uelle;&
Et déja ſon courroux ſemble s'eſtre adouci,&
Depuis qu'elle a pris ſoin de nous rejoindre ici.&
Qui m'euſt dit, qu'vn riuage à mes vœux ſi funeſte,&
Préſenteroit d'abord Pylade aux yeux d'Oreſte,&
Qu'apres plus de ſix mois que ie t'auois perdu,&
A la Cour de Pyrrhus tu me ſerois rendu!\&

\stanza[\enonciateur{Pylade}]
I'en rends graces au Ciel, qui m'arreſtant ſans ceſſe,&
Sembloit m'auoir fermé le chemin de la Gréce,&
Depuis le jour fatal que la fureur des Eaux,&
Preſque aux yeux de Mycéne, écarta nos Vaiſſeaux.&
Combien dans cét exil ay-je ſouffert d'allarmes?&
Combien à vos malheurs ay-je donné de larmes?&
Craignant toûjours pour vous quelque nouueau danger&
Que ma triſte Amitié ne pouuoit partager.&
Sur tout ie redoutois cette Mélancolie&
Où j'ay veu ſi long-temps voſtre Ame enſeuelie.&
Ie craignois que le Ciel, par vn cruel ſecours,&
Ne vous offrît la mort, que vous cherchiez toûjours.&
Mais ie vous voy, Seigneur, \ampersand\  ſi j'oſe le dire,&
Vn Deſtin plus heureux vous conduit en .&
Le pompeux Appareil qui ſuit icy vos pas,&
N'eſt point d'vn Malheureux qui cherche le trépas.\&

\stanza[\enonciateur{Oreste}]
Helas! qui peut ſçavoir le Deſtin qui m'ameine?&
L'Amour me fait icy chercher vne Inhumaine.&
Mais qui ſçait ce qu'il doit ordonner de mon Sort,&
Et ſi ie viens chercher, ou la vie, ou la mort?\&

\stanza[\enonciateur{Pylade}]
Quoy! voſtre Ame à l'Amour, en Eſclaue aſſeruie,&
Se repoſe ſur luy du ſoin de voſtre vie?&
Par quels charmes, apres tant de tourmens ſoufferts&
Peut-il vous inuiter à rentrer dans ſes fers?&
Penſez-vous qu'Hermionne, à inéxorable,&
Vous prépare en vn Sort plus fauorable?&
Honteux d'auoir pouſsé tant de vœux ſuperflus,&
Vous l'abhorriez. Enfin, vous ne m'en parliez plus.&
Vous me trompiez, Seigneur.\&

\stanza[\enonciateur{Oreste}]
\antilabe Ie me trompois moy-méme.&
Amy, n'inſulte point vn Malheureux qui t'aime.&
T'ay-je iamais caché mon cœur \ampersand\  mes deſirs?&
Tu vis naiſtre ma flâme \ampersand\  mes premiers ſoûpirs.&
Enfin, quand Menelas diſpoſa de&
En faueur de Pyrrhus, vangeur de ſa Famille;&
Tu vis mon deſeſpoir, \ampersand\ tu m’as veu depuis&
Traîner de Mers en Mers ma chaîne \ampersand\ mes ennuis.&
Ie te vis à regret, en cét eſtat funeſte,&
Preſt à ſuiure par tout le déplorable Oreſte,&
Toûjours de ma fureur interrompre le cours,&
Et de moy-meſme enfin me ſauuer tous les jours.&
Mais quand ie me ſouuins, que parmy tant d’al-larmes&
Hermionne à Pyrrhus prodiguoit tous ſes charmes,&
Tu ſçais de quel courroux mon cœur alors épris&
Voulut, en l’oubliant, vanger tous ſes mépris.&
Ie fis croire, \ampersand\ ie crûs ma victoire certaine.&
Ie pris tous mes tranſports pour des tranſports de haine;&
Déteſtant ſes rigueurs, rabaiſſant ſes attraits,&
Ie défiois ſes yeux de me troubler iamais.&
Voila comme ie crûs étouffer ma tendreſſe.&
Dans ce calme trompeur j’arriuay dans la Gréce;&
Et ie trouuay d’abord ſes Princes raſſemblez,&
Qu’vn péril aſſez grand ſembloit auoir troublez.&
I’y courus. Ie penſay que la Guerre, \ampersand\ la Gloire,&
De ſoins plus importans rempliroient ma memoire;&
Que mes ſens reprenant leur premiere vigueur,&
L’Amour acheueroit de ſortir de mon Cœur.&
Mais admire auec moy le Sort, dont la pourſuite&
Me fait courir moy-meſme au piege que j’éuite.&
I’entens de tous coſtez qu’on menace Pyrrhus.&
Toute la Gréce éclate en murmures confus.&
On ſe plaint, qu’oubliant ſon Sang, \ampersand\  ſa promeſſe,&
Il éleue en ſa Cour l’Ennemy de la Gréce,&
Aſtyanax, d'Hector jeune \ampersand\  malheureux Fils,&
Reſte de tant de Roys ſous enſeuelis.&
I’apprens, que pour rauir ſon enfance au Suplice,&
Andromaque trompa l’ingénieux Vlyſſe,&
Tandis qu’vn autre Enfant arraché de ſes bras,&
Sous le nom de , fut conduit au trépas.&
On dit, que peu ſenſible aux charmes d’Hermionne,&
Mon Riual porte ailleurs ſon Cœur \ampersand\  ſa Couronne;&
Ménelas, ſans le croire, en paroiſt affligé,&
Et ſe plaint d’vn Hymen ſi long-temps negligé.&
Parmy les déplaiſirs où ſon Ame ſe noye,&
Il s’éleue en la mienne vne ſecrette joye.&
Ie triomphe; \ampersand\  pourtant ie me flate d’abord&
Que la ſeule vengeance excite ce tranſport.&
Mais l’Ingrate en mō Cœur reprit bientoſt ſa place,&
De mes feux mal éteints ie reconnus la trace,&
Ie ſentis que ma haine alloit finir ſon cours,&
Ou plûtoſt ie ſentis que ie l’aimois toûjours.&
Ainſi de tous les Grecs ie brigue le ſuffrage.&
On m’enuoye à Pyrrhus. I’entreprens ce voyage.&
Ie viens voir ſi l’on peut arracher de ſes bras&
Cét Enfant, dont la vie allarme tant d’Eſtats.&
Heureux, ſi ie pouuois dans l’ardeur qui me preſſe,&
Au lieu d’Aſtyanax, luy rauir ma Princeſſe.&
Car enfin n’attens pas que mes feux redoublez,&
Des périls les plus grands, puiſſent eſtre troublez.&
Puis qu’apres tant d’efforts ma reſiſtance eſt vaine,&
Ie me liure en aueugle au tranſport qui m’entraîne,&
I’aime, ie viens chercher Hermionne en ces lieux,&
La fléchir, l’enleuer, ou mourir à ſes yeux.&
Toy qui connois Pyrrhus, que penſes-tu qu’il faſſe?&
Dans ſa Cour, dans ſon Cœur, dy-moy ce qui ſe paſſe.&
Mon Hermionne encor le tient-elle aſſeruy?&
Me rendra-t'il, Pylade, vn Cœurqu’il m’a rauy?\&

\stanza[\enonciateur{Pylade}]
Ie vous abuſerois, ſi i’oſois vous promettre&
Qu’entre vos mains, Seigneur, il voulut la remettre.&
Non, que de ſa Conqueſte il paroiſſe flaté.&
Pour la Veuue d’Hector ſes feux ont éclaté.&
Il l’aime. Mais enfin cette Veuue inhumaine&
N’a payé jusqu’icy ſon amour que de haine,&
Et chaque jour encore on luy voit tout tenter,&
Pour fléchir ſa Captive, ou pour l’épouuanter.&
Il luy cache ſon Fils, il menaſſe ſa teſte,&
Et fait couler des pleurs, qu’auſſi-toſt il arreſte.&
Hermionne elle-meſme a veu plus de cent fois&
Cet Amant irrité reuenir ſous ſes loix,&
Et de ſes vœux troublez luy rapportant l’hommage,&
Soûpirer à ſes pieds moins d’amour, que de rage.&
Ainſi n’attendez pas, que l’on puiſſe aujourd’huy&
Vous répondre d’vn Cœur, ſi peu maiſtre de luy.&
Il peut, Seigneur, il peut dans ce deſordre extré-me,&
Epouſer ce qu’il hait, \ampersand\  perdre ce qu’il aime.\&

\stanza[\enonciateur{Oreste}]
Mais dy-moy, de quel œil Hermionne peut voir&
Ses attraits offenſez, \ampersand\  ſes yeux ſans pouuoir.\&

\stanza[\enonciateur{Pylade}]
Hermionne, Seigneur, au moins en apparance,&
Semble de ſon Amant dédaigner l’inconſtance,&
Et croit que trop heureux d’appaiſerſa rigueur,&
Il la viendra preſſer de reprendre ſon Cœur.&
Mais ie l’ay veuë enfin me confier ſes larmes.&
Elle pleure en ſecret le mépris de ſes charmes.&
Toûjours preſte à partir, \ampersand\ demeurant toûjours,&
Quelquefois elle appelle Oreſte à ſon ſecours.\&

\stanza[\enonciateur{Pylade}]
Ah! ſi ie le croyois, i’irois bientoſt, Pylade,&
Me jetter....\&

\stanza[\enonciateur{Pylade}]
\antilabe Acheuez, Seigneur, voſtre Ambaſſade.&
Vous attendez le Roy. Parlez, \ampersand\  luy montrez&
Contre le Fils d’Hector tous les Grecs conjurez.&
Loin de leur accorder ce Fils de ſa Maiſtreſſe,&
Leur haine ne fera qu’irriter ſa tendreſſe.&
Plus on les veut broüiller, plus on va les vnir.&
Preſſez. Demandez tout, pour ne rien obtenir.&
Il vient.\&

\stanza[\enonciateur{Oreste}]
\antilabe Hé bien, va donc diſpoſer la Cruelle&
A reuoir vn Amant qui ne vient que pour elle.\&
\endnumbering
\end{document}
